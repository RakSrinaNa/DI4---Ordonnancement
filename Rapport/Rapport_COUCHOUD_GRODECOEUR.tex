\documentclass[hideweeklyreports]{polytech/polytech}
\usepackage{lmodern}
\usepackage{textcomp}

\usepackage{graphicx}
\usepackage{float}
\usepackage{pict2e}
\usepackage[justification=centering]{caption}

\newcommand{\img}[3]{%
	\begin{figure}[H]
		\centering
   		\includegraphics[scale=#3]{Images/#1}
  	 	\caption{#2}
	\end{figure}
}
\newcommand{\imgr}[4]{%
   \begin{figure}[H]
      \centering
         \includegraphics[scale=#3,rotate=#4]{Images/#1}
         \caption{#2}
   \end{figure}
}

\schooldepartment{di}
\typereport{ppgldi4}
\reportyear{2017-2018}
\title{Passage d'un algorithme Tabou du langage Python au langage C}
\subtitle{Algorithme Tabou en C}
\student{Thomas}{Couchoud}{thomas.couchoud@etu.univ-tours.fr}
\student{Clément}{Grodecoeur}{clement.grodecoeur@etu.univ-tours.fr}
\academicsupervisor{Jean-Charles}{Billaut}{jean-charles.billaut@univ-tours.fr}


%TODO
\resume{Résumé FR}
\motcle{Mots clé}

\abstract{Abstract EN}
\keyword{Keywords}

\begin{document}
	\chapter{Contexte de la réalisation} %TODO
		
	\chapter{Spécifications du logiciel} %TODO
		%Etude python
		
	\chapter{Modélisation du logiciel} %TODO : Clément
		\section{Structure générale — UML}
			Le projet est séparé en deux grosses parties: production et tests.
			Le code de production se trouve dans src/prod et le code de tests dans src/unit.
			
			Chacun d'entre eux contiennent les fichiers .c ainsi qu'un dossier headers contenant les fichiers .h.
			
			\img{UML.png}{Liens entre les différents fichiers/structures}{0.45}

		\section{Rôles des différents éléments} %TODO : Clément
			Compte tenu de l'ampleur de ce projet, il est important de découper correctement le code en plusieurs éléments ; chacun d'eux a un rôle bien défini. On distinguera 3 catégories dans lesquelles s'inscrivent les structures de données et leurs fonctions : les instances, les solutions et le tabou.
			
			\subsection{Instance}
				Une instance d'un problème à résoudre est représentée par la structure Instance. Cette dernière contient le nombre de tâches et de machines, une liste de tâches ainsi qu'une matrice des distances entre chaque tâche (pour la livraison). Les tâches sont, elles, représentées par la structure Task contenant une liste des temps de production par machine, ainsi qu'une date de livraison. On se réfère ensuite aux différentes tâches en utilisant leur indice dans la liste.
				
			\subsection{Solution}
				Chaque solution admissible à notre problème est représentée par la structure Solution. Celle-ci comporte le nombre de batchs de livraison, une liste contenant ces derniers ainsi que les informations de la solution calculée ; notons qu'une Solution seule ne décrit que l'ordre et le contenu des batchs, et non l'ordre interne de chacun. Ceci est le rôle de la structure SolutionInfo : elle contient l'ordre de chaque batch l'ordre de production des tâches ainsi qu'une liste des temps auxquels chaque tâche est prête à être livrée.
				
			\subsection{Tabou}
				
			\subsection{Autres}
		
	\chapter{Implémentation \& tests} %TODO : Clément
		\section{Implémentation} %TODO : Clément
			\subsection{Transfert du Python au C} %TODO : Clément
			\subsection{Implémentation d'un système de cache des solutions}
				Lors de la traduction du Python au C, nous avons pu remarquer que les appels à la fonction \textbf{eval()} était très fréquent. De plus cette fonction effectue un travail assez lourd ce qui peut facilement prendre une place importante dans le temps de calcul.
				
				Nous avons donc imaginé un système de cache qui garde une solution calculée (\textbf{SolutionInfo}) en mémoire. De cette manière nous évitons des temps de calculs inutiles.
				
				\img{PythonProfilerTop.png}{Profiler exécuté sur le python}{0.3}
				
				Après l'implémentation du cache, et avec un test sur une petite instance, on a pu remarquer que le temps passé sur le calcul du score est le même avec ou sans cache, à savoir environ $43\%$.
				
				Cela est surement du au fait que les packs traités pour cette instance sont de taille $\leq 3$ donc le calcul de l'ordonnancement reste assez simple. Faire les tests sur une instance plus grande serait sûrement plus intéressant. %TODO : Test bigger instance?
			
			\subsection{Diversification aléatoire} %TODO : Clément
			\subsection{Calcul différent de l'itinéraire}
				Le calcul de l'ordre de production joue un rôle important mais l'ordre de livraison aussi. Afin de pouvoir tester différents résultats nous avons décider 2 façons de calculer l'ordre de livraison.
				
				La première méthode est celui transposé depuis le Python, et consiste à livrer dans l'ordre des dates dues. On peut se dire que cette méthode est optimale. Cependant il ne faut pas oublier les temps de voyage entre les points de livraison. Par exemple si on a deux zones A et B qui sont très très loin l'une de l'autre et que les dates dues nous font livrer par alternance A et B, on va passer la plus part du temps sur la route. De manière instinctive on aurait plus tendance à livrer zone par zone.
				
				C'est exactement ce que fait la seconde méthode. Cette fois-ci on livre les produits en allant toujours au plus proche voisin. Cette méthode n'est pas la plus optimale non plus: certes elle diminue les temps de trajet mais elle ne prends plus en compte du tout les dates dues.
				
				Il faut donc essayer les différents algorithmes afin de savoir le quel est le plus adapté au problème courant.
				
				On peut envisager de faire un algorithme qui ferai le mixe des deux: limiter les trajets tout en tenant compte des dates dues. Cependant on peut se poser la question si faire cela est vraiment rentable. En effet nous pourrons éventuellement réduire les temps de retard, mais cela sera au count d'un algorithme plus complexe et qui par conséquent sera moins rapide. 
				
			\subsection{Simplification de l'implémentation des méthodes de recherche}%TODO : Thomas
		
		\section{Tests}
			Des tests unitaires ont été effectués sur l'ensemble des éléments de base du projet. En effet tester la partie algorithme Tabou est difficile puisque cela dépend du nombre d'itérations, des paramètres utilisés, etc. Sachant que la partie Tabou utilise principalement des fonctions implémentés ailleurs, nous nous sommes concentrés dessus.
			
			Ces derniers restent assez simplistes et ne couvrent sûrement pas tout les cas mais ils permettent d'identifier des erreurs contraignantes pour la suite de manière rapide.
			
			Nous allons maintenant détailler un peu plus les tests effectués. Par la suite nous appellerons constructeur les fonctions xxx\_create(), et destructeur les fonctions xxx\_destroy().
			\subsection{Task}
				\begin{itemize}
					\item Vérification du constructeur/destructeur.
					\item Get/Set des dates dues.
					\item Get/Set des temps machines.
				\end{itemize}
				
			\subsection{Instance}
				\begin{itemize}
					\item Vérification du constructeur/destructeur.
					\item Obtention des tâches par date due.
				\end{itemize}
				
			\subsection{Parser}
				\begin{itemize}
					\item Test des valeurs des instances parsées depuis plusieurs fichiers.
				\end{itemize}
				
			\subsection{Pack}
				\begin{itemize}
					\item Vérification du constructeur/destructeur.
					\item Add/Remove/Has tâche.
					\item Switch tâche.
					\item Move tâche.
				\end{itemize}
				
			\subsection{SolutionInfo}
				\begin{itemize}
					\item Vérification du constructeur/destructeur.
					\item Test de l'ordonnancement de production.
					\item Test de l'ordonnancement de livraison.
				\end{itemize}
				
			\subsection{Solution}
				\begin{itemize}
					\item Vérification du constructeur/destructeur.
					\item Copie.
					\item Add/Remove/Has tâche.
					\item Switch tâche.
					\item Move tâche.
				\end{itemize}
				
			\subsection{Sequencer}
				\begin{itemize}
					\item Test des dates de fin de production pour des pack avec des tâches dans un ordre donné.
					\item Test de l'ordonnancement de production pour un pack.
					\item Test du retard de livraison pour des pack avec des tâches dans un ordre donné.
					\item Test de l'ordonnancement de livraison pour un pack.
				\end{itemize}
				
			\subsection{Sort}
				\begin{itemize}
					\item Test du SWAP.
					\item Test d'EBSR.
					\item Test d'EFSR.
				\end{itemize}
				
			\subsection{TabuList}
				\begin{itemize}
					\item Vérification du constructeur/destructeur.
					\item Test de l'ajout/suppression items tabou.
					\item Test si un élément est dans la liste tabou.
					\item Test de clear.
				\end{itemize}
				
			\subsection{SearchResult}
				\begin{itemize}
					\item Vérification du constructeur/destructeur.
				\end{itemize}
		
	\chapter{Résultats obtenus}
		\section{Comparaison avec le python} %TODO : Thomas
		\section{Comparaison avec l'implémentation en codage direct} %TODO
	
	\chapter{Autres}
		Ce chapitre concerne différents mécanismes qui ont pu être mis en place tout au long du projet ainsi que différentes améliorations possibles au programme.
		
		\section{Parallélisme} %TODO : Clément
		
		\section{Fichier de configuration}
			La mise en place d'un fichier de configuration serait une grande avancée. En effet, cela permettrai d'éviter la recompilation du projet après la modification d'un paramètre.
			
			Cependant cela nous empêcherai d'utiliser des commandes pré-processeur ce qui nous forcerai à faire un bon nombre de conditions if qui auraient toujours la même valeur tout au long de l'exécution.
		
		\section{Github — Travis}
			Nous avons tout au long du projet utilisé un gestionnaire de version (VCS): Git. Ce dernier nous a permit de pouvoir travailler de manière collaborative de manière simplifiée. En effet chacun pouvait se focuser sur une partie précise du projet et ainsi paralléliser un peu le développement.
			
			La gestion des branches sur ce dernier était très simple. Nous faisions notre développement sur une branche dev, et lorsque ce qui avait été réalisé marchait, nous le mergions dans la branche master afin d'avoir une sorte de point de sauvegarde.
			
			Enfin, nous avons ajouté en plus du VCS un outil d'intégration continue (CI) afin de vérifier nos tests, lancer Valgrind, et effectuer des comparaisons au Python. De cette manière nous pouvions nous concentrer plus efficacement sur le développement. Les tests ne sont pas oubliés de cette manière et le feedback sur les tests était fait de manière transparente.
			\img{TravisAll.png}{Aperçu général des builds}{0.5}
			\img{TravisCompare.png}{Résultats des comparaisons}{0.575}
			
			Afin de pouvoir lancer ces comparaisons, un utilitaire en JAVA a été créé. Ce dernier prend une liste d'instance en paramètres et effectue les actions suivantes:
			\begin{itemize}
				\item Execution des instances sur le programme en C et en Python.
				\item Récupération des résultats.
				\item Vérification de la cohérence du score avec la solution fournie.
				\item Comparaison des deux scores.
			\end{itemize}
			
\end{document}